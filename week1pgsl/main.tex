\documentclass{article}[14pt]
\usepackage{graphicx} % Required for inserting images
\usepackage{titlesec}
\usepackage{kantlipsum}
\usepackage{tabularx}


\titleformat{\section}
  {\normalfont\Large\bfseries}{\thesection}{1em}{}[{\titlerule[0.8pt]}]
  
\title{\textbf{Resume}}
\author{\textbf{Pradeep Dalal}}
\date{July 2023}


\begin{document}
\maketitle
\section{Education}

{\large \textbf{Master of Technology(Artificial Intelligence)}}\\
{\normalsize Indian Institute Of Technology, Ropar}

 \vspace{10px}
 
{\raggedleft \large \textbf{Bachelor of Technology(Computer Science Engineering)}}\\
{\normalsize Indore Institute of Science and Technology}
{\small \null\hfill CGPA- 9.20}

 \vspace{10px}

 
{\raggedleft \large \textbf{Senior Secondary Education}}\\
{\normalsize ST. Theresa Senior School}
{\small \null\hfill Percentage- 94.20}

\vspace{10px}


{\raggedleft \large \textbf{Higher Secondary Education}}\\
{\normalsize ST. Theresa Senior School}
{\small \null\hfill CGPA- 9.2}

\section{Projects}

   \subsection{Automatic Timetable Generator} 
    \begin{itemize}
        \item {\normalsize A web based approach to solve the university scheduling problem}
        \item {\normalsize Created an automated time table scheduler with the help of genteic algorithm which generates schedule automatically from the given inputs and resolves conflict automatically}

        \item {\normalsize Role: Backend Developer}
        \item {\normalsize Technology Used - HTML, CSS, JavaScript, JSP, Servlet}
    \end{itemize}

    
    \subsection{College Interaction System}
    \begin{itemize}
        \item {\normalsize A web based solution for the college staff and students to interact with each other}
        
        \item {\normalsize Students can communicate with faculty in a personalised manner and can submit the assignments and faculty can also assign assignments and checks the status of the students}

        \item {\normalsize Role: Backend Developer}
        \item {\normalsize Technology Used - HTML, CSS, JavaScript, JSP, Servlet}
    \end{itemize}
    
\section{Programming Languages}

\begin{enumerate}
    \item {\textbf {C/C++}}\cite{dos1996textual}
    \item {\textbf {Java}}\cite{haggan2004research}
    \item {\textbf {HTML/CSS}}
\end{enumerate}


\section{Technical Subjects}

\begin{enumerate}
    \item {\textbf {Object Oriented Programming}}
    \item {\textbf {Operating System}}
    \item {\textbf {Computer Network}}
    \item {\textbf {Data Structure and Algorithm}}
\end{enumerate}

\section{Certificates}

\begin{enumerate}
    \item { \textbf{Great Learning Certificate} - Java Programming}
    \item{\textbf{Great Learning Certificate} - Object Oriented Programming in Java}
    \item{\textbf{Udemy Certificate} - Web Development}

\end{enumerate}

\begin{tabularx}{0.8\textwidth} { 
  | >{\raggedright\arraybackslash}X 
  | >{\centering\arraybackslash}X 
  | >{\raggedleft\arraybackslash}X | }
 \hline
 item 11 & item 12 & item 13 \\
 \hline
 item 21  & item 22  & item 23  \\
\hline
\end{tabularx}

\bibliographystyle{plain}
\bibliography{bibliography}

\end{document}
